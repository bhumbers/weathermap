\documentclass{article} % For LaTeX2e
\usepackage{nips13submit_e,times}
\usepackage{hyperref}
\usepackage{url}
%\documentstyle[nips13submit_09,times,art10]{article} % For LaTeX 2.09


\title{Predicting Meteorological Values on a Spatial Grid}


\author{
Felipe Hern\'{a}ndez \\
\texttt{felipeh@andrew.cmu.edu} \\
\And
Ben Humberston\\
\texttt{bhumbers@cs.cmu.edu} \\
}

\newcommand{\fix}{\marginpar{FIX}}
\newcommand{\new}{\marginpar{NEW}}

\nipsfinalcopy

\begin{document}

\maketitle

\begin{abstract}
TODO
\end{abstract}

\section{Introduction}
\label{sec:intro}

\subsection{Motivation}
\label{sec:motivation}
Weather forecasts are broadly used; from personal activity planning (What clothes should I wear 
today? Will it be a good time to plan outdoor events?); to large-scale economical decisionmaking (What activities should be prioritized on a crop next week? What precautions should the 
air-traffic controllers enforce on a given day?); to emergency preparation and response (What 
alternate routes should become available due to snow?, Where should the emergency vehicles 
be sent during a flood event?). The availability and accuracy of forecasts thus have a profound 
impact on human activities at many levels, both in measurable and unmeasurable aspects.
However, predicting weather is a difficult research problem. Most often, physically-based 
models with global scale are used to forecast future conditions. In this project, we will instead 
take a machine learning approach focused on a local scale and attempt to predict atmospheric 
variables at a specific geographic location. The forecasts will be based on prior atmospheric 
states in the neighborhood of the selected location. In particular, we will attempt to predict 
variables such as pressure, precipitation, and temperature based on the previous values of 
these variables based on regular historical snapshots produced from satellite observations by 
NASA.

\subsection{Related Work}\label{sec:related_work}
TODO

\section{Method}
\label{sec:method}
TODO

Citation test: \cite{Durban2001}.

\bibliographystyle{unsrt}
\bibliography{10_701_Project_Refs}

\end{document}